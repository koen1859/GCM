\subsection{Damaged Goods}
Suppose that a monopolist offers a good with quality $q_2$. The monopolist can also "damage" the good it produces. By doing so it produces a quality
$q_1<q_2$. Marginal costs for the damaged goods are $c_1>c_2$. The higher cost is because of the damaging.

Assume there are two types of consumers, with valuations $\theta_iV(q)$, where $\theta_2>\theta_1$. Also, assume a proportion $\lambda$ of the
population has type 1. We make two additional assumptions:
\begin{enumerate}
	\item When the monopolist only sells the high-end product, it would find it profitable to
	      serve only the type 2 consumers.
	\item When the monopolist would only sell the damaged good, it would sell find it most
	      profitable to sell that to both types of consumers.
\end{enumerate}

Consider assumption 1. When it chooses to serve only the high types, the profit-maximizing price would be $p_2=\theta_2V(q_2)$. Profits then equal:
\begin{align}
	\Pi_H=(1-\lambda)(\theta_2V(q_2)-c_2).
\end{align}
Suppose the monopolist would serve both consumers. Then it would set $p_2=\theta_1V(q_2)$, profits then equal:
\begin{align}
	\Pi=\theta_1V(q_2)-c_2.
\end{align}
The assumption that only the high types are served boils down to:
\begin{align}
	(1-\lambda)(\theta_2V(q_2)-c_2)                     & >\theta_1V(q_2)-c_2 \\
	\Longrightarrow(\theta_1-(1-\lambda)\theta_2)V(q_2) & <\lambda c_2.
\end{align}

Now consider the case that the monopolist only produces the damaged good. When only serving high types, the profit-maximizing price would be
$p_1=\theta_2V(q_1)$. Profits then equal:
\begin{align}
	\Pi_D=(1-\lambda)(\theta_2V(q_1)).
\end{align}
Suppose it would serve both consumers, the price is then $p_1=\theta_1V(q_1)$, with profits:
\begin{align}
	\Pi=\theta_1V(q_1)-c_1.
\end{align}
The assumption that he would serve both types thus boils down to assuming that:
\begin{align}
	\theta_1V(q_1)-c_1>(1-\lambda)(\theta_2V(q_1))
	\label{eq:chdamage}
\end{align}
We are now able to tackle the problem of the monopolist that produces both goods.
It then has to set $p_1$ and $p_2$ as to maximize:
\begin{align}
	\Pi_{HD}=\lambda(p_1-c_1)+(1-\lambda)(p_2-c_2).
\end{align}
Note that the difference between this problem and the one
discussed earlier, is that the qualities $q_1$ and $q_2$ are exogenously given.
With the same
arguments as before, we can show that in equilibrium the individual rationality constraint
of type 1, and the incentive compatibility constraint of type 2 have to bind:
\begin{align}
	p_1 & =\theta_1V(q_1)               \\
	p_2 & =\theta_2(V(q_2)-V(q_1))+p_1.
\end{align}
The monopolist's profits are thus given by:
\begin{align}
	\Pi_{HD}=\Pi_H+(\theta_1V(q_1)-c_1)-(1-\lambda)(\theta_2V(q_1)-c_1).
\end{align}
Using (\ref{eq:chdamage}) this implies that $\Pi_{HD}>\Pi_H$.
Hence, offering both the high-quality and the damaged goods strictly increases the monopolist’s profits. The types 2 will also be strictly
better off with the introduction of the
damaged good: in the case without damaged goods, they were kept at zero surplus. Now,
they earn a strictly positive surplus, for the usual reasons. In this simplified specification,
the types 1 are indifferent between whether the damaged good is introduced: in
both cases, their net surplus is zero.
In Deneckere and McAfee (1996), the low types are
strictly better off. This is due to the fact that they assume that each individual has a
downward sloping demand curve for both qualities. In such a case, by setting a single
price for each quality, the monopolist is never able to capture the entire consumer surplus. Therefore, with two products, the types 1 are left with some consumer surplus as
well. Note that this result would disappear if the monopolist would simply sell packages
of the high-quality and the damaged good.
