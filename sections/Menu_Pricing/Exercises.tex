\subsection{Exercises}
\begin{enumerate}
	\item There are two types of consumers. High types have a willingness to pay $\theta\sqrt{q}$ for a
	      product of quality $q$, with $\theta > 1$, low types have willingness to pay $\sqrt{q}$. Both types
	      occur in equal numbers. A monopolist is able to produce two qualities $q_1 = 9$ and
	      $q_2 = 36$. Production costs are 2 per unit, regardless of the quality produced.
	      \begin{enumerate}
		      \item Suppose the monopolist can only produce the high quality product. Determine
		            the price that it will charge and the profits it will make (note: this will depend
		            on $\theta$). Do the same in case the monopolist can only produce the low quality
		            product.
		      \item Suppose the monopolist can produce both qualities. Determine the profit-
		            maximizing prices, and resulting profits. For what values of $\theta$ will the monopolist choose to produce both qualities?
		      \item Now suppose that the monopolist can choose the quality of the low quality
		            product, whereas the quality of the high-quality product is still fixed at 36.
		            Determine what low quality he is going to set. Explain.
	      \end{enumerate}

	      Solution:
	      \begin{enumerate}
		      \item With quality $q_2=36$, low type has willingness to pay $\sqrt{36}=6$.
		            For the high type $6\theta$. The monopolist can choose to set price s.t.
		            only the high type buys or everyone buys. Doing the first, set price to $6\theta$,
		            then profits are $3\theta-1$. Doing the second leads to profits of 4. This is higher if
		            $\theta<5/3$.

		            With quality $q_1=9$, willingness to pay for low type is 3, for high type 3$\theta$.
		            The same analysis as before yields that the monopolist sets price 3 if $\theta<4/3$
		            and price $3\theta$ else.
		      \item Using the result from the lecture notes, the prices are:
		            \begin{align*}
			            p_1 & =\sqrt9                                \\
			            p_2 & =\theta(\sqrt{36}-\sqrt9)+3=3\theta+3.
		            \end{align*}
		            This leads to profits:
		            \begin{align*}
			            \frac{6+3\theta}{2}-2=\frac{3}{2}\theta+1
		            \end{align*}
		            This is never higher than only selling the low quality. Only selling the high quality
		            leads to profits $\max\{3\theta-1,4\}$, which is always higher than when selling both
		            qualities.
		      \item Let $x^2$ be the quality of the low quality product. When offering both products,
		            the monopolist sets prices:
		            \begin{align*}
			            p_1 & =x              \\
			            p_2 & =x+\theta(6-x),
		            \end{align*}
		            which leads to profits:
		            \begin{align*}
			            \frac{2x+\theta(6-x)}{2}-2=\frac{1}{2}x(2-\theta)+3\theta-2.
		            \end{align*}
		            For $\theta<2$ this is increasing in $x$, which means that the monopolist will set
		            $x=6$ (so quality is $6^2=36$). This means that the monopolist only sells the high quality
		            product to both types. For $\theta>2$, this is decreasing in $x$ so the monopolist will set $x=0$.
		            This means it only sells high quality to high type.
	      \end{enumerate}
	\item A monopolist sells cars. She faces two types of consumers. The utility a consumer
	      obtains from the car is solely determined by the expected amount of kilometers
	      that the car is able to drive. Denote this expected kilometrage as $\E{k}$, denoted
	      in units of 100.000 kilometer. Type 1 consumers have a utility of owning the car
	      that equals $3\sqrt{\E{k}}$. Type 2 consumers have a utility of owning the car that equals
	      $4\sqrt{\E{k}}$. The fraction of type 1 consumers in the population is 1/2. The fraction
	      of type 2 consumers is also 1/2. Initially, the monopolist produces cars that run
	      100,000 kilometers for sure, so $\E{k}$ = 1. The costs of producing such a car are 1.
	      But the monopolist realizes that she is also able to produce a car that is able to run
	      for 100,000 kilometers with probability 1/2, and will not run at all (so 0 kilometers)
	      with probability 1/2 as well. Consumers can perfectly observe this. The costs of
	      producing such a low-quality car are denoted by $c_L$.
	      \begin{enumerate}
		      \item Determine for which values of $c_L$ it is profitable for the monopolist to also
		            supply the low-quality car, given that she can charge prices for both types of
		            cars in a profit-maximizing manner.
		      \item Now suppose that $c_L = 1$ and the monopolist can choose the expected kilometrage of the cheap car to be any value below 100,000.
		            Which value will he
		            choose? Again, of course, we assume that after he has chosen this value, he
		            can charge prices for both types of cars in a profit-maximizing manner.
	      \end{enumerate}

	      Solution:
	      \begin{enumerate}
		      \item Suppose the monopolist sells both cars. Then the maximization problem is:
		            \begin{align*}
			            \max\,       & \frac{1}{2}(P_H-1)+\frac{1}{2}(P_L-c_L)          \\
			            \text{s.t. } & \begin{cases}
				                           3\sqrt{1/2}-P_L\geq0             & \text{(IR-1)} \\
				                           4\sqrt{1}-P_H\geq0               & \text{(IR-2)} \\
				                           3\sqrt{1/2}-P_L\geq3\sqrt{1}-P_H & \text{(IC-1)} \\
				                           4\sqrt{1}-P_H\geq4\sqrt{1/2}-P_L & \text{(IC-2)}
			                           \end{cases}
		            \end{align*}
		            From the same argument as the notes we have that (IR-1) and (IC-2) are binding, hence:
		            \begin{align*}
			            P_L   & =\sqrt{1/2}      \\
			            4-P_H & =3\sqrt{1/2}-P_L \\
			            \Longrightarrow          \\
			            P_L   & =3\sqrt{1/2}     \\
			            P_H   & =4-\sqrt{1/2}.
		            \end{align*}
		            Then, profits are:
		            \begin{align*}
			            \Pi & =\frac{1}{2}\br{4-\sqrt{1/2}-1}+\frac{1}{2}\br{3\sqrt{1/2}-c_L} \\
			                & =\frac{1}{2}\sqrt2 +\frac{3}{2}-\frac{1}{2}c_L.
		            \end{align*}
		            Suppose the monopolist sells only high quality car. Only selling to high types implies
		            $P=4$, so $\Pi=\frac{1}{2}(4-1)=3/2$. Selling to both gives $P=3$ so
		            $\Pi=3-1=2$, hence in this case the monopolist would sell to both. Also supplying low quality car
		            is profitable if
		            \begin{align*}
			            \frac{1}{2}\sqrt2 +\frac{3}{2}-\frac{1}{2}c_L & >2         \\
			            \Longrightarrow c_L                           & <\sqrt2-1.
		            \end{align*}
		      \item Let $\gamma$ be the quality of the low-quality car. Suppose the monopolist
		            offers both types of cars. Then the maximization problem is:
		            \begin{align*}
			            \max\,       & \frac{1}{2}(P_H-1)+\frac{1}{2}(P_L-1)               \\
			            \text{s.t. } & \begin{cases}
				                           3\sqrt{\gamma}-P_L\geq0             & \text{(IR-1)} \\
				                           4\sqrt{1}-P_H\geq0                  & \text{(IR-2)} \\
				                           3\sqrt{\gamma}-P_L\geq3\sqrt{1}-P_H & \text{(IC-1)} \\
				                           4\sqrt{1}-P_H\geq4\sqrt{\gamma}-P_L & \text{(IC-2)}
			                           \end{cases}
		            \end{align*}
		            (IR-1) and (IC-2) binding implies:
		            \begin{align*}
			            P_L   & =3\sqrt{\gamma}     \\
			            4-P_H & =4\sqrt{\gamma}-P_L \\
			            \Longrightarrow             \\
			            P_L   & =3\sqrt\gamma       \\
			            P_H   & =4-\sqrt\gamma
		            \end{align*}
		            Then, the profits are:
		            \begin{align*}
			            \Pi & =\frac{1}{2}\br{4-\sqrt\gamma-1}+\frac{1}{2}\br{3\sqrt\gamma-1} \\
			                & =\sqrt\gamma+1.
		            \end{align*}
		            This is strictly increasing in $\gamma$ and thus $\gamma=1$. This implies
		            the monopolist only sells the high quality car.
	      \end{enumerate}
\end{enumerate}
