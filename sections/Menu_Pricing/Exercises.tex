\subsection{Exercises}
\begin{enumerate}
	\item There are two types of consumers. High types have a willingness to pay $\theta\sqrt{q}$ for a
	      product of quality $q$, with $\theta > 1$, low types have willingness to pay $\sqrt{q}$. Both types
	      occur in equal numbers. A monopolist is able to produce two qualities $q_1 = 9$ and
	      $q_2 = 36$. Production costs are 2 per unit, regardless of the quality produced.
	      \begin{enumerate}
		      \item Suppose the monopolist can only produce the high quality product. Determine
		            the price that it will charge and the profits it will make (note: this will depend
		            on $\theta$). Do the same in case the monopolist can only produce the low quality
		            product.
		      \item Suppose the monopolist can produce both qualities. Determine the profit-
		            maximizing prices, and resulting profits. For what values of $\theta$ will the monopolist choose to produce both qualities?
		      \item Now suppose that the monopolist can choose the quality of the low quality
		            product, whereas the quality of the high-quality product is still fixed at 36.
		            Determine what low quality he is going to set. Explain.
	      \end{enumerate}

	      Solution:
	\item A monopolist sells cars. She faces two types of consumers. The utility a consumer
	      obtains from the car is solely determined by the expected amount of kilometers
	      that the car is able to drive. Denote this expected kilometrage as $\E{k}$, denoted
	      in units of 100.000 kilometer. Type 1 consumers have a utility of owning the car
	      that equals $3\sqrt{\E{k}}$. Type 2 consumers have a utility of owning the car that equals
	      $4\sqrt{\E{k}}$. The fraction of type 1 consumers in the population is 1/2. The fraction
	      of type 2 consumers is also 1/2. Initially, the monopolist produces cars that run
	      100,000 kilometers for sure, so $\E{k}$ = 1. The costs of producing such a car are 1.
	      But the monopolist realizes that she is also able to produce a car that is able to run
	      for 100,000 kilometers with probability 1/2, and will not run at all (so 0 kilometers)
	      with probability 1/2 as well. Consumers can perfectly observe this. The costs of
	      producing such a low-quality car are denoted by $c_L$.
	      \begin{enumerate}
		      \item Determine for which values of $c_L$ it is profitable for the monopolist to also
		            supply the low-quality car, given that she can charge prices for both types of
		            cars in a profit-maximizing manner.
		      \item Now suppose that $c_L = 1$ and the monopolist can choose the expected kilometrage of the cheap car to be any value below 100,000.
		            Which value will he
		            choose? Again, of course, we assume that after he has chosen this value, he
		            can charge prices for both types of cars in a profit-maximizing manner.
	      \end{enumerate}

	      Solution:
	      \begin{enumerate}
		      \item
		      \item
	      \end{enumerate}
\end{enumerate}
