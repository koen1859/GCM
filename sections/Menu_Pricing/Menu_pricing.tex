\subsection{Menu pricing}
A monopolist wants to price discriminate, but there is no exogenous signal
of each customer's demand function. In other words: the monopolist is unable
to divide the customers into groups purely on the basis of some exogenous variable
such as age. Some consumers have a higher elasticity of demand than others, but the monopolist
is not able to tell which is which. The monopolist can offer a menu of bundles
where the consumers can choose from.
\subsubsection{The Model}
Assume that the consumers have a surplus of consuming quality $q$ of a good that
equals $U(\theta, q)$, where $\theta$ is a parameter that differs per consumer.
The cost of providing quality $q$  is $c(q)$. We make the following assumptions:
\begin{enumerate}
	\item $U(\theta,0)=0$
	\item $\pder{U}{q}>0$
	\item $\spder{U}{q}<0$
	\item $c'\geq0$
	\item $c''\geq0$
\end{enumerate}

We assume that there are two types of consumers, with taste parameters
$\theta_1$, $\theta_2$, with $\theta_1<\theta_2$. The proportion of type-1
consumers is $\lambda$, hence $(1-\lambda)$ type 2. The costs of providing
a product with quality $q$ is given by $c(q)$, with $c'>0$ and $c''\geq0$.

Suppose the monopolist offers two packages. One is denoted $(p_1, q_1)$, the other
$(p_2,q_2)$. We assume the following single-crossing condition is satisfied:
\begin{align}
	U(\theta_2,q_2)-U(\theta_2,q_1)\geq U(\theta_1,q_2)-U(\theta_1,q_1)
\end{align}
A high type is always willing to pay more for any quality upgrade than a low
type is.

The monopolist maximizes its profits:
\begin{align}
	\underset{p_1,q_1,p_2,q_2}{\max} & \lambda(p_1-c(q_1))+(1-\lambda)(p_2-c(q_1)) \\
	\text{s.t.}                      & \begin{cases}
		                                   U(\theta_1,q_1)-p_1\geq
		                                   U(\theta_1,q_2)-p_2      & \text{(IC-1)} \\
		                                   U(\theta_2,q_2)-p_2\geq
		                                   U(\theta_2,q_1)-p_1      & \text{(IC-2)} \\
		                                   U(\theta_1,q_1)-p_1\geq0 & \text{(IR-1)} \\
		                                   U(\theta_2,q_2)-p_2\geq0 & \text{(IR-2)}
	                                   \end{cases}
\end{align}
(IC-1), (IC-2) require that the net utility of buying the bundle designed for
you is at least as high as buying the other bundle. (IR-1), (IR-2) require that
you get positive utility from buying the bundle designed for you.
\subsubsection{Benchmark: complete information}
Consider the case in which the monopolist knows the types of the consumers
it faces. Hence, it does not need to worry about a type 1 choosing bundle 2
and vice-versa, simply because a type 1 consumer only has the possibility to
choose a type 1 bundle, and a type 2 only can choose bundle 2. This means that
the incentive constraints do not have to be satisfied.

The monopolist now designs bundle $i$ to maximize:
\begin{align}
	\underset{p_i,q_i}{\max}\, & p_i-c(q_i)                \\
	\text{s.t. }               & U(\theta_i,q_i)-p_i\geq0.
\end{align}
In the profit-maximizing solution the constraint is binding. Otherwise the
monopolist can increase $p_i$ to get more profit. Then the problem becomes:
\begin{align}
	\underset{q_i}{\max}\,U(\theta_i,q_i)-c(q_i),
	\label{eq:ch4max}
\end{align}
which implies that the optimal solution is given by:
\begin{align}
	\pder{U(\theta_i,q_i)}{q_i} & = \der{c(q_i)}{q_i}, \\
	p_i                         & = U(\theta_i,q_i).
\end{align}
The consumer is indifferent between buying the bundle and not buying, in this
case.

For a given bundle $q_i$, the monopolist always maximizes its profits by
charging an amount that equals a consumer's willingness to pay for this bundle,
so $p_i=U_i(\theta_i,q_i)$. Then the profit-maximizing bundle to offer to type
$i$ consumers is the one where marginal cost equals marginal utility.
\subsubsection{Social optimum}
Now consider social welfare. As always, social welfare equals the firm's
profits plus the consumer surplus. For a consumer of type $i$, this implies:
\begin{align}
	\text{SW}_i=(p-c(q_i))+(U(\theta_i,q_i)-p)=U(\theta_i,q_i)-c(q_i).
\end{align}
This is exactly (\ref{eq:ch4max}). Hence, the monopoly will maximize social
welfare.
\subsubsection{Solving the model with incomplete information}
