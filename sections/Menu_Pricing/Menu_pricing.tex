\subsection{Menu pricing}
A monopolist wants to price discriminate, but there is no exogenous signal
of each customer's demand function. In other words: the monopolist is unable
to divide the customers into groups purely on the basis of some exogenous variable
such as age. Some consumers have a higher elasticity of demand than others, but the monopolist
is not able to tell which is which. The monopolist can offer a menu of bundles
where the consumers can choose from.
\subsubsection{The Model}
Assume that the consumers have a surplus of consuming quality $q$ of a good that
equals $U(\theta, q)$, where $\theta$ is a parameter that differs per consumer.
The cost of providing quality $q$  is $c(q)$. We make the following assumptions:
\begin{enumerate}
	\item $U(\theta,0)=0$
	\item $\pder{U}{q}>0$
	\item $\spder{U}{q}<0$
	\item $c'\geq0$
	\item $c''\geq0$
\end{enumerate}
