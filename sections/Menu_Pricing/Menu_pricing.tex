\subsection{Menu pricing}
A monopolist wants to price discriminate, but there is no exogenous signal
of each customer's demand function. In other words: the monopolist is unable
to divide the customers into groups purely on the basis of some exogenous variable
such as age. Some consumers have a higher elasticity of demand than others, but the monopolist
is not able to tell which is which. The monopolist can offer a menu of bundles
where the consumers can choose from.
\subsubsection{The Model}
Assume that the consumers have a surplus of consuming quality $q$ of a good that
equals $U(\theta, q)$, where $\theta$ is a parameter that differs per consumer.
The cost of providing quality $q$  is $c(q)$. We make the following assumptions:
\begin{enumerate}
	\item $U(\theta,0)=0$
	\item $\pder{U}{q}>0$
	\item $\spder{U}{q}<0$
	\item $c'\geq0$
	\item $c''\geq0$
\end{enumerate}

We assume that there are two types of consumers, with taste parameters
$\theta_1$, $\theta_2$, with $\theta_1<\theta_2$. The proportion of type-1
consumers is $\lambda$, hence $(1-\lambda)$ type 2. The costs of providing
a product with quality $q$ is given by $c(q)$, with $c'>0$ and $c''\geq0$.

Suppose the monopolist offers two packages. One is denoted $(p_1, q_1)$, the other
$(p_2,q_2)$. We assume the following single-crossing condition is satisfied:
\begin{align}
	U(\theta_2,q_2)-U(\theta_2,q_1)\geq U(\theta_1,q_2)-U(\theta_1,q_1)
\end{align}
A high type is always willing to pay more for any quality upgrade than a low
type is.

The monopolist maximizes its profits:
\begin{align}
	\underset{p_1,q_1,p_2,q_2}{\max} & \lambda(p_1-c(q_1))+(1-\lambda)(p_2-c(q_1)) \\
	\text{s.t.}                      & \begin{cases}
		                                   U(\theta_1,q_1)-p_1\geq
		                                   U(\theta_1,q_2)-p_2      & \text{(IC-1)} \\
		                                   U(\theta_2,q_2)-p_2\geq
		                                   U(\theta_2,q_1)-p_1      & \text{(IC-2)} \\
		                                   U(\theta_1,q_1)-p_1\geq0 & \text{(IR-1)} \\
		                                   U(\theta_2,q_2)-p_2\geq0 & \text{(IR-2)}
	                                   \end{cases}
	\label{eq:ch4maxprob}
\end{align}
(IC-1), (IC-2) require that the net utility of buying the bundle designed for
you is at least as high as buying the other bundle. (IR-1), (IR-2) require that
you get positive utility from buying the bundle designed for you.
\subsubsection{Benchmark: complete information}
Consider the case in which the monopolist knows the types of the consumers
it faces. Hence, it does not need to worry about a type 1 choosing bundle 2
and vice-versa, simply because a type 1 consumer only has the possibility to
choose a type 1 bundle, and a type 2 only can choose bundle 2. This means that
the incentive constraints do not have to be satisfied.

The monopolist now designs bundle $i$ to maximize:
\begin{align}
	\underset{p_i,q_i}{\max}\, & p_i-c(q_i)                \\
	\text{s.t. }               & U(\theta_i,q_i)-p_i\geq0.
\end{align}
In the profit-maximizing solution the constraint is binding. Otherwise the
monopolist can increase $p_i$ to get more profit. Then the problem becomes:
\begin{align}
	\underset{q_i}{\max}\,U(\theta_i,q_i)-c(q_i),
	\label{eq:ch4max}
\end{align}
which implies that the optimal solution is given by:
\begin{align}
	\pder{U(\theta_i,q_i)}{q_i} & = \der{c(q_i)}{q_i}, \\
	p_i                         & = U(\theta_i,q_i).
\end{align}
The consumer is indifferent between buying the bundle and not buying, in this
case.

For a given bundle $q_i$, the monopolist always maximizes its profits by
charging an amount that equals a consumer's willingness to pay for this bundle,
so $p_i=U_i(\theta_i,q_i)$. Then the profit-maximizing bundle to offer to type
$i$ consumers is the one where marginal cost equals marginal utility.
\subsubsection{Social optimum}
Now consider social welfare. As always, social welfare equals the firm's
profits plus the consumer surplus. For a consumer of type $i$, this implies:
\begin{align}
	\text{SW}_i=(p-c(q_i))+(U(\theta_i,q_i)-p)=U(\theta_i,q_i)-c(q_i).
\end{align}
This is exactly (\ref{eq:ch4max}). Hence, the monopoly will maximize social
welfare.
\subsubsection{Solving the model with incomplete information}
We now solve (\ref{eq:ch4maxprob}) with incomplete information. We do not have the assumption that the monopolist can observe the type of a
consumer, and can offer that consumer only one package. Now we assume that that the monopolist needs to offer both types of packages to the consumers,
and hope that each consumer will buy the correct one.

Note that we have:
\begin{align}
	U(\theta_2,q_2)-p_2\geq U(\theta_2,q_1)-p>U(\theta_1,q_1)-p_1.
	\label{eq:ch4eqchain}
\end{align}
The first inequality is simply (IC-2), the second inequality then follows from $\theta_2 > \theta_1$. Suppose that (IR-1) is not binding. Then,
(\ref{eq:ch4eqchain}) implies that (IR-2) is also not binding. Then, the monopolist could just increase $p_1$ and $p_2$ and increase its profits.
Hence, at the profit-maximizing solution, we must have (IR-1) binding. By (\ref{eq:ch4eqchain}) we then have that (IR-2) is strictly satisfied.
This means that (IR-2) can be removed from the list of restrictions, since it is implied by (IR-1).

Suppose that (IC-2) is not binding. We then have:
\begin{align}
	U(\theta_2,q_2)-p_2>U(\theta_2,q_1)-p_1>U(\theta_1,q_1)-p_1=0.
\end{align}
This implies that increasing $p_2$ by some small value $\epsilon$ does not violate (IR-2). Hence this cannot be profit-maximizing. At the
profit-maximizing solution, (IC-2) is binding.

Add (IC-1) and (IC-2) to find:
\begin{align}
	U(\theta_2,q_2)-U(\theta_2,q_1)\geq U(\theta_1,q_2)-U(\theta_1,q_1).
\end{align}
This is exactly the single-crossing solution, provided that $q_2\geq q_1$. Hence, at the profit-maximizing solution, $q_2\geq q_1$. The fact that
(IC-2) is binding implies:
\begin{align}
	p_2-p_1=U(\theta_2,q_2)-U(\theta_2,q_1)>U(\theta_1,q_2)-U(\theta_1,q_1),
\end{align}
using $\theta_2>\theta_1$, $q_2\geq q_1$ and the single crossing condition. This implies:
\begin{align}
	U(\theta_1,q_2)-p_2<U(\theta_1,q_1)-p_1.
\end{align}
Hence, if (IC-2) is binding, (IC-1) is strictly satisfied. This means we can delete (IC-1) from the list of restrictions. We can write the
monopolist's problem as:
\begin{align}
	\underset{q_1,q_2}{\max}\, & \lambda(p_1-c(q_1))+(1-\lambda)(p_2-c(q_2)) \\
	\text{s.t. }               & \begin{cases}
		                             p_2=U(\theta_2,q_2)-U(\theta_2,q_1)+p_1 \\
		                             p_1=U(\theta_1,q_1)
	                             \end{cases}
\end{align}
or
\begin{align}
	\underset{q_1,q_2}{\max}\,\Pi(q_1,q_2) & =\lambda(U(\theta_1,q_1)-c(q_1))                                      \\
	                                       & +(1-\lambda)(U(\theta_2,q_2)-U(\theta_2,q_1)+U(\theta_1,q_1)-c(q_2)).
\end{align}
FOC:
\begin{align}
	\pder{\Pi}{q_1}                            & =\pder{U}{q_1}(\theta_1,q_1)-(1-\lambda)\pder{U}{q_1}(\theta_2,q_1)-\lambda\pder{c}{q_1}(q_1)=0, \\
	\pder{\Pi}{q_2}                            & =(1-\lambda)\br{\pder{U}{q_2}(\theta_2,q_2)-\pder{c}{q_2}(q_2)}=0                                \\
	\Longrightarrow\pder{U}{q_1}(\theta_1,q_1) & =\lambda\pder{c}{q_1}(q_1)+(1-\lambda)\pder{U}{q_1}(\theta_2,q_1),                               \\
	\pder{U}{q_2}(\theta_2,q_2)                & =\pder{c}{q_2}(q_2).
\end{align}
Note that $q_2>q_1$. Hence, as $\spder{U}{q}(\theta_2,q)<0$, we have
$\pder{U}{\theta_2,q_1}>\pder{U}{q_2}(\theta_2,q_2)=\pder{c}{q_2}(q_2)\geq\pder{c}{q_1}(q_1)$. This implies that
$\lambda\pder{c}{q_1}(q_1)+(1-\lambda)\pder{U}{q_1}(\theta_2,q_1)>\pder{c}{q_1}(q_1)$, so the optimal solution has
$\pder{U}{q_1}(\theta_1,q_1)>\pder{c}{q_1}(q_1)$ (marginal cost is smaller than marginal utility).

There are several things to note:
\begin{enumerate}
	\item Note that (4.22) implies that the type 2 consumes exactly the amount that he would
	      also choose to consume would he face a price $c$. Hence, the type 2 consumes the
	      amount that is socially optimal. This is a general result in these types of models:
	      there is \textit{no distortion at the top}.
	\item Also note from that the type 2 gets a strictly positive surplus: with
	      $U (\theta_2, q_2) > p_2$, a type 2 pays less for package 2 than he would be willing to pay.
	      The difference is type 2’s informational rent: if the monopolist would know his
	      type, he would not get any surplus. Hence, the surplus is due to type 2 having
	      private information regarding his type.
	\item From (IR-1) binding, we have that a type 1 pays exactly the amount he is willing to pay
	      for $q_1$ units of the product. Hence, the lowest type does not obtain any utility from
	      consuming the good: there is no surplus at the bottom.
	\item Under the assumption that, in equilibrium, both types are served (which was
	      made throughout the entire analysis), we have that $\pder{U}{q_1}(\theta_1, q_1)>\pder{c}{q_1} (q_1)$. With
	      $\spder{U}{q_1} (\theta_1, q_1) < 0$, this implies that type 1 consumes less than his socially optimal
	      amount. In fact, from Lemma 3, we have that the package $(p_1, q_1)$ is designed such
	      that a type 2 is exactly indifferent between consuming $(p_1, q_1)$ and $(p_2, q_2)$. This is
	      also a general result: in a model with more types, we always have that a type is
	      indifferent between consuming his own package and the package designed for the
	      next-highest type. In a sense, this also holds for the lowest type: he is indifferent
	      between consuming the package designed for him, and consuming nothing.
	\item Note that the rhs of (4.21) is decreasing in $\lambda$, again conditional on both types being
	      served. This implies that as $\lambda$, the fraction of types 1 in the population, increases,
	      the package designed for type 1 moves in the direction of his socially optimal package. Also, as
	      $\theta_2$ and $\theta_1$ move closer together, the same is true.
	\item Finally, note that throughout the analysis we assume that the monopolist finds it
	      profitable to offer two different packages to the two different types of consumers.
	      We solved the model under that restriction. However, that may not always be the
	      case. Depending on the parameters, the monopolist may find it more profitable
	      to only offer one package. For example, if there are relative few low types, then a
	      monopolist may find it more profitable to only offer a package aimed at the high
	      type, and not selling anything to the low types. In that case, it can charge a higher
	      price to the high types, but has to give up any profits it was making on the low
	      types.
\end{enumerate}
If we would implement the complete information equilibrium in the incomplete information case, the incentive compatibility constraint of the
high type would not be satisfied: he would prefer consuming bundle $q_1$ and obtain utility $U(\theta_2,q_1)-p_1>0$ rather than consuming bundle
$q_2$ and obtain utility $U(\theta_2,q_2)-p_2=0$.

In equilibrium, the monopolist decreases the price for bundle 2 and decrease $q_1$. Both actions make it less attractive for a type 2 to choose
the bundle of a type 1. In the new optimum, the low type pays exactly what he is willing to pay, whereas the high type is indifferent between
consuming his own bundle or that of a low type, i.e. $\pder{U}{q_2}(\theta_2,q_2)-P_2=\pder{U}{q_1}(\theta_2,q_1)-P_1$.
\textit{What the company is trying to do is prevent
	the passengers who can pay the second-class fare from traveling third-class; it hits the
	poor, not because it wants to hurt them, but to frighten the rich.}
