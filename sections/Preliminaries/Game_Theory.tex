\subsection{Game Theory}
\subsubsection{Nash Equilibrium}
Each player is choosing the best possible strategy given the strategies chosen by the other players.
\begin{align}
	s^*_i\in \underset{s_i}{\argmax}\,U_i(s^*_1,\cdots,s^*_i,\cdots,s_n^*),\forall i=1,\cdots n.
\end{align}
There is $\in$ instead of $=$ since this Nash Equilibrium does not need to be unique.
\subsubsection{Reaction functions}
A reaction function (or best reply or best response function) gives the best action for a player
given the actions of the other players. The Nash Equilibrium is where all reaction functions intersect.
\subsubsection{Symmetry}
When the game is symmetric, i.e. all players are in the same conditions, have the same reaction function
etcetera, then the players are anonymous (They are indistinguishable from each other except for
name or index). Then all players choose the same strategy $s^*$ in the Nash Equilibrium.
\subsubsection{Sub-game perfect equilibrium}
When players do not move simultaneously, i.e. the players move sequentially, we need to refine
definition of Nash Equilibrium: the sub-game perfect equilibrium requires that the strategy profile
under consideration is not only the equilibrium for the entire game but also for each sub-game.
\subsubsection{Moves of nature}
Many models in IO involve uncertainty. This is often modeled as a “move of nature” in a
multistage game. The moment at which the uncertainty is resolved, is referred to as the
move of nature. Such games can again be solved using backward induction
*** Candidate equilibrium
For ease of exposition, we will often use the concept of a candidate equilibrium. In many
models, it is possible to make an educated guess as to what the equilibrium might be. We
will refer to such an educated guess as a candidate equilibrium. A candidate equilibrium thus is a
strategy profile that may be a (subgame perfect) Nash equilibrium, but for
which we still have to check whether that really is the case. This approach for finding
an equilibrium is often easier than deriving an equilibrium from scratch. Note however
that if it turns out that the candidate equilibrium is a true equilibrium, we still have not
established whether that equilibrium is unique.
*** Mixed strategies
Players are not restricted to always play some action $a_i^*$ in equilibrium. A mixed strategy
equilibrium has each player drawing its action from some probability distribution $F_i(a_i)$,
defined on some domain $A_i$. Given the strategies played by all other players, each player $i$
is indifferent between the actions among which it mixes. Hence, $\E{U_i(a_i)}$ is constant
for all $a_i\in A_i$.
