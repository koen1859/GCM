\subsection{Exercises}
\begin{enumerate}
	\item Consider the following Cournot model. Two firms set quantities. Demand is given by
	      $q=1-p$. Marginal costs are either equal to 0 or 0.4, both with equal probability.
	      Derive the Cournot equilibrium if
	      \begin{enumerate}
		      \item uncertainty is resolved before firms set their quantities.
		      \item uncertainty is resolved after firms set their quantities.
	      \end{enumerate}

	      Solution:
	      \begin{enumerate}
		      \item The game is symmetric, hence in equilibrium both firms set the same
		            price.
		            \begin{align*}
			            \pi_i=(1-q_1-q_2-c)q_i\Longrightarrow\frac{\partial\pi_i}{\partial q_i} & =1-q_j-2q_i-c      \\
			            \overset{\text{FOC}}{\Longrightarrow} q_i                               & =\frac{1-q_j-c}{2} \\
			            \text{Imposing symmetry: }                                              & q=\frac{1-c}{3}
		            \end{align*}
		            Then $c=0$ gives $q=\frac{1}{3}$, $c=0.4$ gives $q=\frac{1}{5}$.
		      \item The expected marginal cost is $\E{c}=0.2$. We can fill this in
		            the previous formula:
		            \begin{align*}
			            q=\frac{1-\E{c}}{3}=\frac{0.8}{3}
		            \end{align*}
	      \end{enumerate}
	\item Consider a Hotelling model. Consumers are uniformly distributed on a line of unit
	      length. Consumers either buy one unit of the good or none at all. Each consumer
	      obtains gross utility $v$ from consuming the product. We have two firms: one is
	      located at 0, the other is located at 1. Marginal costs for both firms are constant
	      and equal to $c$. However, transportation costs are constant: a consumer that has
	      to travel a distance $x$ incurs transport costs $tx^2$. Derive the equilibrium prices.

	      Solution: This game is symmetric hence in equilibrium both firms set the same price.
	      Consider the indifferent consumer with location $z$, then:
	      \begin{align*}
		      v-P_0-tz^2                   & =v-P_1-t(1-z)^2                 \\
		      \Longleftrightarrow P_0+tz^2 & =P_1+t(1-z)^2                   \\
		      \Longrightarrow z            & =\frac{1}{2}+\frac{P_1-P_0}{2t}
	      \end{align*}
	      Firm 0 its expected profits are:
	      \begin{align*}
		      \pi_0                                             & =(P_0-c)z=(P_0-c)\frac{1}{2}+\frac{P_1-P_0}{2t} \\
		      \Longrightarrow\frac{\partial\pi_0}{\partial P_0} & =\frac{p_1-p_0+t}{2t}-\frac{p_0-c}{2t}          \\
		      \overset{\text{FOC}}{\Longrightarrow}P_0          & =\frac{1}{2}(P_1+c+t)
	      \end{align*}
	      Imposing symmetry:
	      \begin{align*}
		      P=\frac{1}{2}(P+c+t)\Longleftrightarrow P=c+t
	      \end{align*}
	\item Consider the Perloff-Salop model where consumers have a valuation that is uniformly
	      distributed on $[0, 1]$. For simplicity, $c = 0$. Assume however that $v = 0$ such
	      that not all consumers buy in equilibrium. Derive the equilibrium prices.

	      Solution: In equilibrium, prices of both firms are equal to $p$. A consumer buys if $\max\{\epsilon_A,\epsilon_B\}>p$.
	      This happens with probability:
	      \begin{align*}
		      \P{\max\{\epsilon_A,\epsilon_B\}>p} & =1-\P{\max{\epsilon_A,\epsilon_B}\leq p}    \\
		                                          & =1-\P{\epsilon_A\leq p}\P{\epsilon_B\leq p} \\
		                                          & =1-p^2
	      \end{align*}
	      Hence both firms face demand $q=\frac{1-p^2}{2}$.\\
	      Suppose firm $A$ sets the price slightly higher than firm $B$, $p_A=p+\Delta$. Then, demand is:
	      \begin{align*}
		      q_A=\frac{1}{2}(1-\Delta)^2-\frac{1}{2}p^2
	      \end{align*}
	      with profits:
	      \begin{align*}
		      \pi_A=(p_A-c)q_A
	      \end{align*}
	      FOC gives:
	      \begin{align*}
		      \pi_A=p_A\pder{q_A}{p_A}+q_A & =0             \\
		      \text{with }\pder{q_A}{p_A}  & =-(1-\Delta)   \\
		      \Longleftrightarrow q_A      & =(1-\Delta)p_A
	      \end{align*}
	      Imposing symmetry:
	      \begin{align*}
		      p & =\frac{1}{2}-\frac{p^2}{2} \\
		      p & =\sqrt{2}-1
	      \end{align*}
\end{enumerate}
