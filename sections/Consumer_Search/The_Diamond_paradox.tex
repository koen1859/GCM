\subsection{The Diamond paradox}
\subsubsection{Model}
Consider a slight change of the basic model with homogeneous products and equal
marginal costs. We assume that consumers have to incur search costs to
find out which price each firm charges. These search costs are $s$,
where $s$ can be arbitrarily small. Visiting the first shop is free,
but to visit any additional shop, a consumer has to incur search costs
$s$. Suppose again that there is a unit mass of consumers that all have
a willingness to pay for one unit of the product that equals $v$. Firms
have constant marginal costs $c < v$. It is crucial that consumers
cannot observe prices before they visit a shop. For simplicity, we assume
that there are only two firms.

The timing of the game is as follows. First, firms set prices that are
not observable. Second, consumers visit the shop of their first choice,
and observe the price that is set by that shop. Third, consumers either
buy from this shop, decide to visit the other shop, or decide not to buy
at all. The equilibrium now also depends on what consumers expect. That
makes it hard to derive reaction functions in the usual manner; these
reaction functions now also depend on the behavior of consumers. We
therefore choose a different manner of analysis. We first postulate a
candidate equilibrium. Then, we derive whether any firm (or the
consumers, for that matter), has an incentive to defect from that
equilibrium.

A natural candidate equilibrium seems to be the one that is the
equilibrium in the standard Bertrand model, in which both firms charge a
price $c$, consumers split evenly in the choice of their first shop,
and consumers also decide to buy at the shop they first visit. It is easy
to see, however, that this cannot be an equilibrium. Suppose that firm 1
deviates and increases its price to $c + \frac{s}{2}$. First, such a
defection cannot influence the number of consumers that visit this shop,
as consumers can only observe the price after they have visited. Hence,
all consumers that visit this shop are now unpleasantly surprised. Yet,
they still have no reason to choose a different strategy in the sub-game
that follows. Going to the other shop implies a cost-saving of
$\frac{s}{2}$ due to a lower price (note that we consider a deviation
from the Nash equilibrium by firm 1, which implies that firm 2 is still
charging price $c$), but an increase in search cost by $s$. Hence,
consumers will not continue search and the deviation is profitable.

But the same argument holds for any symmetric candidate equilibrium
that has $p<v$. For a price $p\geq v$, consumers also will not switch for
a small price increase, but a firm is not willing to defect such a
manner, as by definition it would decrease its profits.

Thus, the unique equilibrium has both firms charging the monopoly price,
even with infinitesimally small search costs. This is known as the
\textit{Diamond paradox}. Also note that this result does not hinge on
the number of firms. For any $n\geq2$, the analysis applies and the
unique equilibrium has prices equal to the monopoly price.
\subsubsection{What if the first visit is not free?}
In this model, the assumption that the first visit is free is somewhat odd. Suppose that
a consumer would also have to pay s to visit the first firm. Following the logic above,
the unique Nash equilibrium would have all firms charging the monopoly price $pm = v$.
But now suppose that consumers also have to pay s to visit the first firm. This will not
change the logic of the analysis; it would still be an equilibrium to have $p = v$. Once the
consumer has visited the first firm, the search costs $s$ are already sunk so she could either
buy at that firm and pay 1, or go home without the product and still having incurred
search costs $s$.

But a rational consumer will foresee this. Knowing that all firms will charge price $v$,
and knowing that she has to incur search costs $s$ to visit one, she knows that it is not a
good idea to enter this market, since doing so so will leave her with a negative surplus
of $-s$. Firms would like to commit to charge a lower price of $v-s$, but cannot do so;
once a consumer enters their shop they have an incentive to defect to charging a price $v$
anyhow. Hence, the market will break down and cease to exist.

We can get around this by assuming that each consumer has an individual downward
sloping demand function $D(p)$. At a monopoly price of $p_m$, each consumer then demands
more than one unit and still has some consumer surplus $S$. Firms do not have an incentive
to defect to a higher price: Again consumers will not switch for a small price increase,
but a firm is not willing to defect such a manner, as by definition it would decrease its
profits. As long as its consumer surplus $S$ this is bigger than the search cost $s$, the market
would still exist, as the consumer would still be willing to pay the first visit.
\subsubsection{Implications}
This result, that even infinitessimally small search costs are enough to switch from an
equilibrium with marginal cost pricing to one with monopoly pricing, is known as the
\textit{Diamond paradox}. It shows that even the slightest perturbation of the original model is
already enough to end up in the other extreme outcome.

Another surprising outcome of the model is that, although we have introduced search
costs, people do not actually end up searching in the equilibrium of this model: in
equilibrium, they also end up buying from the first firm they encounter.

To find a way out of the Diamond paradox, we can do two things. First, we can
assume that not all consumers have positive search costs. If some consumers can observe
all prices, firms may still have an incentive to defect from the monopoly price. Second,
we may assume that products are differentiated. We will discuss these two approaches in
what follows. But before we can do so, we first have to delve somewhat deeper in optimal
consumer search.
