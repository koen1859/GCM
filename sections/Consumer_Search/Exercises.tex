\subsection{Exercises}
\begin{enumerate}
	\item Consider Varian's model of sales as described in Section 2.3.
	      Suppose the number of firms is $n>2$. Derive the mixed strategy
	      equilibrium.

	      Solution:
	\item Consider the following variation of Varian (1980)’s model of sales.
	      A duopoly
	      competes in prices. Costs of production are zero. There is a mass 1 of consumers
	      that is willing to pay at most 2, and a mass 1 of consumers that is willing to pay at
	      most 3. For both groups there is a fraction $\lambda$ that is informed and a fraction $1-\lambda$
	      that is uninformed.
	      \begin{enumerate}
		      \item Derive the best-reply function of each firm.
		      \item Derive the Nash equilibrium in prices.
	      \end{enumerate}

	      Solution:
	      \begin{enumerate}
		      \item Setting $p=2$ is always more profitable than setting price $p>2$ for either firm, regardless of the choice
		            of the other firm. Since even if $p_2=0$ (i.e. all informed customers go to firm 2) then still $p_1=2$ gives
		            profit $\pi_1=2(1-\lambda)$ while $p_1=3$ gives profits $\pi_1=\frac{3(1-\lambda)}{2}$, which is clearly smaller.
		            This difference increases even further if $p_2\geq2$.\\
		            Suppose that $p_2=2$. Then there are 3 choices for firm 1:
		            \begin{itemize}
			            \item $p_1=2\Longrightarrow q_1=1\Longrightarrow \pi_1=2 $
			            \item $p_1=p_2-\epsilon$ with $\epsilon$ arbitrarily small. This gives demand
			                  $q_1=2\lambda+(1-\lambda)=1+\lambda\Longrightarrow\pi_1=2(1+\lambda ) $
		            \end{itemize}
		            It is clear that the third choice is the best option.\\
		            Suppose $p_2<2$. Then there are 2 choices for firm 1:
		            \begin{itemize}
			            \item $p_1=2\Longrightarrow q_1=1-\lambda\Longrightarrow\pi_1=2(1-\lambda)$
			            \item $p_1=p_2-\epsilon$ with $\epsilon$ arbitrarily small. This gives demand
			                  $q_1=2\lambda+(1-\lambda)=1+\lambda\Longrightarrow\pi_1=p_2(1+\lambda)$
		            \end{itemize}
		            The first option is optimal if
		            $p_2(1+\lambda)\leq2(1-\lambda)\Longleftrightarrow p_2\leq\frac{2(1-\lambda)}{1+\lambda} $. This yields the
		            best-reply function:
		            \begin{align*}
			            p_1=\begin{cases}
				                2            & \text{if }p_2\leq\frac{2(1-\lambda)}{1+\lambda} \\
				                p_2-\epsilon & \text{otherwise}
			                \end{cases}
		            \end{align*}
		      \item There is no equilibrium in pure strategies, we need to look for a mixed-strategy equilibrium on the interval
		            $[\underline{p},\overline{p}]$, where we now from the previous part that $\overline{p}=2$.\\
		            If a firm sets $p=2$ its price is the highest for sure. This gives profits $2(1-\lambda )$.
		            A firm that sets price $p$ will sell $1+\lambda$ if this price is the lowest. Otherwise, it will only
		            sell $1-\lambda$. Then the expected profits are:
		            \begin{align*}
			            F(p)(1-\lambda)p+(1-F(p))(1+\lambda)p=p+p\lambda(1-2F(p))
		            \end{align*}
		            For a mixed strategy equilibrium, this expected profit needs to equal the profit of $p=2$. Hence:
		            \begin{align*}
			            F(p)=\frac{p-2+\lambda(p+2)}{2p\lambda}
		            \end{align*}
		            We can find lower bound $\underline{p}$ by solving $F(p)=0$.
		            Hence, $\underline p = \frac{2(1-\lambda)}{1+\lambda}$.
	      \end{enumerate}
	\item Consider a model of search with differentiated products as analyzed in this chapter.
	      There are two firms that each have zero costs. Match values are uniformly
	      distributed on $[0, 1]$. A share $\lambda$ of consumers have search costs $s = 0.25$. The other
	      $1-\lambda$ have search costs $s=0.16$. Derive the equilibrium in prices.

	      Solution: Suppose both firms charge price $p$. Let's refer to
	      the low search cost consumers with $L$ and to the
	      high search cost consumers with $H$. We have
	      $\hat\epsilon=1-\sqrt{2s}$, so
	      $\hat\epsilon_L=1-0.4\sqrt{2}$ and
	      $\hat\epsilon_H=1-0.5\sqrt{2}$. Suppose firm $A$ defects to price
	      $p_A=p-\Delta$. Let
	      $\hat\epsilon_A=\lambda\hat\epsilon_H+(1-\lambda)\hat\epsilon_L$.
	      Then (As per the result in ch. 2.6.1):
	      \begin{align*}
		      q_A & = \frac{1}{2}+\frac{1}{2}\Delta(1+\hat\epsilon_A-\Delta)                          \\
		          & = \frac{1}{2}(p-p_A)(1+\hat\epsilon_A-p+p_A)                                      \\
		          & = \frac{1}{2}+\frac{1}{2}\br{p(1+\hat\epsilon_A-p)+p_A(2p-1-\hat\epsilon_A-p_A)}.
	      \end{align*}
	      Profits are $\pi_A=p_Aq_A$, hence the FOC is:
	      \begin{align*}
		      \pder{q_A}{p_A}+q_A=0
	      \end{align*}
	      We know:
	      \begin{align*}
		      \pder{q_A}{p_A}        & = p-p_A-\frac{1}{2}(1+\hat\epsilon_A) \\
		      (\text{with symmetry}) & = -\frac{1}{2}(1+\hat\epsilon_A).
	      \end{align*}
	      With symmetry, $q_A=\frac{1}{2}$. Then fill in into the FOC:
	      \begin{align*}
		      -\frac{1}{2}(1+\hat\epsilon_A)p^*+\frac{1}{2} & =0                           \\
		      \Longrightarrow p^*                           & =\frac{1}{1+\hat\epsilon_A}.
	      \end{align*}
	      $\hat\epsilon_A=\lambda\hat\epsilon_H+(1-\lambda)\hat\epsilon_L
		      =1-\sqrt{2}(0.4+0.1\lambda)$.
	      So the final answer is:
	      \begin{align*}
		      p^*=\frac{1}{2-\sqrt{2}(0.4+0.1\lambda)}
	      \end{align*}
	\item In the model of search with differentiated products described in these notes, we
	      assumed that consumers pick a firm at random to visit first. Let us now assume that
	      that is no longer the case, and can advertise to attract consumers. More precisely,
	      if firm 1 puts out $a_1$ ads and firm 2 puts out $a_2$, then the probability that firm 1
	      will be visited first is given by $a_1/(a_1 + a_2)$. The cost of each ad are $1/4$,
	      Suppose that match values are uniformly distributed on $[0,1]$, and that firms set
	      prices and advertising levels simultaneously. Derive equilibrium profits as a function
	      of search costs $s$.

	      Solution: We now have demand for firm 1:
	      \begin{align*}
		      D_1=\frac{a_1}{a_1+a_2}(1-\hat\epsilon-\Delta)
		      +\frac{a_2}{a_1+a_2}(1-\hat\epsilon-\Delta)\hat\epsilon
		      +\frac{1}{2}\hat\epsilon^2.
	      \end{align*}
	      Fill in $\Delta=p_1-p^*$ and calculate the derivative:
	      \begin{align*}
		      \pder{D_1}{p_1}=-\frac{a_1}{a_1+a_2}-\frac{a_2}{a_1+a_2}\hat\epsilon.
	      \end{align*}
	      Total profits are equal to $\pi_1=p_1D_1-\frac{1}{4}a_1$. Then
	      calculate the FOC's wrt. to $p_1$ and $a_1$:
	      \begin{align*}
		      \pder{\pi_1}{p_1} & =D_1+p_1\pder{D_1}{p_1},                            \\
		      \pder{\pi_1}{a_1} & =\frac{a_2}{(a_1+a_2)^2}(1-\hat\epsilon-\Delta)p_1-
		      \frac{a_2}{(a_1+a_2)^2}(1-\hat\epsilon-\Delta)\hat\epsilon p_1-\frac{1}{4}=0.
	      \end{align*}
	      Then impose symmetry:
	      \begin{align*}
		      \pder{\pi_1}{p_1} & =\frac{1}{2}+p^*\br{-\frac{1}{2}-\frac{1}{2}\hat\epsilon}=0, \\
		      \pder{\pi_1}{a_1} & =\frac{1}{4a}(1-\hat\epsilon)p^*
		      -\frac{1}{4a}(1-\hat\epsilon)\hat\epsilon p^*-\frac{1}{4}=0.
	      \end{align*}
	      Hence:
	      \begin{align*}
		p^*&=\frac{1}{1+\hat\epsilon},\\
		a^*&=(1-\hat\epsilon)^2p^*.
	      \end{align*}
\end{enumerate}
