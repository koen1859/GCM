\subsection{Exercises}
\begin{enumerate}
	\item Consider Varian's model of sales as described in Section 2.3.
	      Suppose the number of firms is $n>2$. Derive the mixed strategy
	      equilibrium.

	      Solution:
	\item Consider the following variation of Varian (1980)’s model of sales.
	      A duopoly
	      competes in prices. Costs of production are zero. There is a mass 1 of consumers
	      that is willing to pay at most 2, and a mass 1 of consumers that is willing to pay at
	      most 3. For both groups there is a fraction $\lambda$ that is informed and a fraction $1-\lambda$
	      that is uninformed.
	      \begin{enumerate}
		      \item Derive the best-reply function of each firm.
		      \item Derive the Nash equilibrium in prices.
	      \end{enumerate}

	      Solution:
	\item Consider a model of search with differentiated products as analyzed in this chapter. 
	      There are two firms that each have zero costs. Match values are uniformly
	      distributed on $[0, 1]$. A share $\lambda$ of consumers have search costs $s = 0.25$. The other
	      $1-\lambda$ have search costs $s=0.16$. Derive the equilibrium in prices.

	      Solution:
	\item In the model of search with differentiated products described in these notes, we
	      assumed that consumers pick a firm at random to visit first. Let us now assume that
	      that is no longer the case, and can advertise to attract consumers. More precisely,
	      if firm 1 puts out $a_1$ ads and firm 2 puts out $a_2$, then the probability that firm 1
	      will be visited first is given by $a_1/(a_1 + a_2)$. The cost of each ad are $1/4$,
	      Suppose that match values are uniformly distributed on $[0,1]$, and that firms set
	      prices and advertising levels simultaneously. Derive equilibrium profits as a function
	      of search costs $s$.

	      Solution:
\end{enumerate}
