\subsection{A model of sales}
Consider a duopoly that competes in prices. Assume there is a mass of
consumers that equals 1, where each consumer is willing to pay at most
1 for the product. Each consumer demands only one unit of product:
\begin{align}
	D(p)=
	\begin{cases}
		1 & \text{if }p\leq1 \\
		0 & \text{if }p>1
	\end{cases}
\end{align}
Assume 0 marginal costs, and that there are two types of consumers:
A fraction of \(\lambda\) consumers is informed, which implies they can
observe the prices of both firms. The remaining fraction of \(1-\lambda\)
is uninformed: they pick at random which firm to go to and buy the
product if the price does not exceed 1.

The expected profits for firm \(i\) are given by:
\begin{align}
	\Pi_i(p_i,p_j)=
	\begin{cases}
		0                                                & \text{if }p_i>1         \\
		\frac{1}{2}(1-\lambda)p_i                        & \text{if }1\geq p_i>p_j \\
		\frac{1}{2}p_i                                   & \text{if }p_i=p_j\leq1  \\
		\left[\lambda+ \frac{1}{2}(1-\lambda) \right]p_i & \text{if }p_i<p_j\leq1
	\end{cases}
\end{align}
provided \(p_i\leq1\).

We can derive the best-response functions. First, note that whenever
firm \(i\) would want to charge a higher price than \(j\), the best
option is to charge a price of 1, since it faces the same demand as any
other price in \((p_j,1]\), and maximizes its profits if it sets its
price at 1. This implies the following. Suppose \(p_j>0\). The best choice
for firm \(i\) is to either slightly undercut \(p_j\) which yields profits
\(\left[\lambda + \frac{1}{2}(1-\lambda) \right]p_j\) or to set \(p_i=1\),
yields profits \(\frac{1-\lambda}{2}\). Firm \(i\) strictly prefers the
first strategy when:
\begin{align}
	\lambda p_j+\frac{1}{2}(1-\lambda) & >\frac{1}{2}(1-\lambda)      \\
	\Longleftrightarrow p_j            & >\frac{1-\lambda}{1+\lambda}
\end{align}
This yields the reaction functions:
\begin{align}
	R_i(p_j)=
	\begin{cases}
		1            & \text{if }p_j\leq\frac{1-\lambda}{1+\lambda} \\
		p_j-\epsilon & \text{if }p_j>\frac{1-\lambda}{1+\lambda}
	\end{cases}
\end{align}
There is no Nash equilibrium in pure strategies. Suppose firm 1 sets
\(p_1=1\). Firm 2 will slightly undercut. But now, firm 1 wants to
undercut firm 2 again. This process continues until
\(p=\frac{1-\lambda}{1+\lambda}\). Then no firm has incentive to
undercut the other, but each firm does have an incentive to defect by
setting its price equal 1, where the whole process starts anew.

There is however, an equilibrium in mixed strategies. Suppose both firms
draw their price from some continuous distribution function \(F(p)\)
on the support \(\left[\underline{p},\overline{p}\right]\). For this too
be an equilibrium, we need that given that firm \(j\) uses the mixed
strategy, any \(p_i\in\left[\underline{p},\overline{p}\right]\) must
yield the same expected profit for firm \(i\).

Note that the profits that firms make in the mixed strategy equilibrium
can not be lower than \(\frac{1-\lambda}{2}\), then a firm could just set
price 1 and have more profits. This implies we can not have \(\overline{p}>1\).
Consider the possibility that \(\overline{p}<1\). A firm charging price
\(\overline{p}\) is certain that it is charging the highest price on the
market. Then it can do better by charging price 1 instead. Hence we must
have \(\overline{p}=1\).

A firm charging \(\overline{p}=1\) makes profit \(\frac{1-\lambda}{2}\),
this implies that any \(p\in\left[\underline{p},\overline{p}\right]\)
must yield those same profits. Consider a firm charging \(\underline{p}\).
With certainty, this firm charges the lowest price. Its profits then
equal \(\left[\lambda+ \frac{1}{2}(1-\lambda) \right]\underline{p}\).
This has to equal \(\frac{1-\lambda}{2}\), we thus have that
\(\underline{p}=\frac{1-\lambda}{1+\lambda}\).

If firm \(i\) charges price \(p\), the probability that firm \(j\) charges
a lower price if \(F(p)\). When setting some
\(p_i\in\left[\underline{p},\overline{p}\right]\), expected profits are:
\begin{align}
	\E{\pi_i(p)} & =F(p)\left(\frac{1}{2}(1-\lambda)\right)p+
	(1-F(p))\left(\lambda+\frac{1}{2}(1-\lambda)\right)p      \\
	             & =\frac{1}{2}(1-\lambda)p+(1-F(p))\lambda p
\end{align}
We thus need:
\begin{align}
	\frac{1}{2}(1-\lambda)p+(1-F(p))\lambda p=\frac{1-\lambda}{2}, &  &
	\forall p\in\left[\frac{1-\lambda}{1+\lambda},1\right]
\end{align}
Solving for \(F(p)\) yields:
\begin{align}
	F(p)=1-\frac{(1-\lambda)(1-p)}{2\lambda p}
\end{align}
The equilibrium profits of both firms in this equilibrium equal
\(\frac{1-\lambda}{2}\), which is strictly positive. Hence, there is no
Bertrand paradox in this set-up.
