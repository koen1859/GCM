\subsection{Search with homogeneous products}
Suppose there are $n$ firms and a unit mass of consumers. Customers have
unit demand and a willingness to pay of 1. A fraction $\lambda$ of
consumers has zero reach costs. The remaining fraction $1-\lambda$ has
reach costs $s\in(0,1)$. Again assume that the first visit is free.

An equilibrium in pure strategies does not exist. Firms either have
incentive to charge a slightly lower price to capture all
the shoppers or a much higher price to make a substantial profit on the
non-shoppers that happen to visit.

Given the reservation price $\hat p$ that non-shoppers will use we can
derive the price distribution $F(p)$ that firms will use.
Given the $F(p)$ that the firms will use we can derive the reservation
price $\hat p$ that consumers will use.

Suppose that one firm charges a price $p>\hat p$. Then both non-shoppers
and shoppers will not buy from this firm, hence this can not be part
of an equilibrium. This implies in equilibrium all firms set $p\leq\hat p$
which implies there will be no search in equilibrium, all non-shoppers
buy from the first firm they encounter. \textit{The equilibrium thus
	necessarily has all shoppers buying from the cheapest firm, and
	all non-shoppers buying from the first firm they encounter.}
