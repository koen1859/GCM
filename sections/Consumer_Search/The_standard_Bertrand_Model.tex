\subsection{The standard Bertrand Model}
Consider two firms that produce identical goods. There are no frictions, i.e
the consumers buy from the firm that sets the lowest price. If they set
the same price they both get half the market share. Assume firms can
fulfill any demand they face. The demand function is $q=D(p)$, marginal
costs constant equal $c$. Monopoly profits assumed to be strictly concave,
bounded and the monopoly price is given by:
\begin{align}
	p_m=\underset{p}{\argmax}(p-c)D(p)
\end{align}
Demand for firm $i$ is written \(D_i(p_i,p_j)\):
\begin{align}
	D_i(p_i,p_j) =
	\begin{cases}
		D(p_i)            & \text{if }p_i<p_j \\
		\frac{1}{2}D(p_i) & \text{if }p_i=p_j \\
		0                 & \text{if }p_i>p_j
	\end{cases}
\end{align}
The profits of firm $i$ are then:
\begin{align}
	\Pi_i(p_i,p_j)=(p_i-c)D_i(p_i,p_j)
\end{align}
This profit function is discontinuous.

First consider the case when \(p_j>p_m\). In such a case the best
reply of firm \(i\) is to set \(p_i=p_m\). Then it attracts all consumers,
while charging the monopoly price.

Now suppose \(c<p_j\leq p_m\). Firm \(i\) has three choices. Charging
price \(p_i>p_j\) it has zero profits. Charging price \(p_i=p_j\),
its sales are \(\frac{1}{2}D(p_j)\). Given the concavity of the profit
function and the fact that \(p_j<p_m\), the best response of \(i\) is to
set the price slightly below \(p_j\), say \(p_i=p_j-\epsilon\), with
\(\epsilon\) arbitrarily small.

Consider the case when \(p_j<c\). Undercutting now yields negative profits.
Hence, the best reply is to set any price \(p_i>p)j\) with zero profit.

Finally, consider the case in which \(p_j=c\). Undercutting again leads
to negative profits. Both \(p_i=p_j\) and \(p_i>p_j\) yield 0 profits.
Therefore the best reply is to set any \(p_i\geq c\).

Summarizing:
\begin{align}
	R_i(p_j)=
	\begin{cases}
		\in(p_j,\infty) & \text{if }p_j<c         \\
		\in[c,\infty)   & \text{if }p_j=c         \\
		p_j-\epsilon    & \text{if }c<p_j\leq p_m \\
		p_m             & \text{if }p_j>p_m
	\end{cases}
\end{align}
From this we conclude there is unique Nash equilibrium where both firms
charge price \(c\), with zero profit for both firms.

This result is known as the \textit{Bertrand paradox}. It qualifies as a paradox
since it is hard to believe that two firms is already enough to restore
competitive outcome in a market. When only two firms are active in a
market, one would expect them to make positive profits.
