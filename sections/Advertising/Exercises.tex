\subsection{Exercises}
\begin{enumerate}
	\item
	      \begin{enumerate}
		      \item Consider the Grossman/Shapiro model as described in the lecture notes, but
		            with $t = 1$ and $c = 0$. However, different from the model in the notes, half of
		            consumers that are not informed by the ads of firm 1, still learn about product
		            1 through word-of-mouth. Something similar holds for firm 2. Write down the
		            profit functions for this case, and derive the system of equations that pin down
		            equilibrium prices and equilibrium numbers of informed consumers (hence do
		            not explicitly solve for those values; that is too much hassle. But do write
		            equilibrium prices in terms of equilibrium fractions of informed consumers and
		            vice versa)
		      \item Suppose we would do the same in a Butters model (that is, assume that
		            half of consumers that are not informed by the ads of firm 1, still learn about
		            product 1 through word-of-mouth). Argue whether the result that the amount
		            of ads are socially optimal would still hold (hence do not derive, only give an
		            intuitive argument).
	      \end{enumerate}

	      Solution:
	      \begin{enumerate}
		      \item The number of consumers that is now informed about firm 1 is:
		            \begin{align*}
			            \phi_i=\Phi_1+\frac{1}{2}(1-\Phi_1)=\frac{1}{2}(1+\Phi_1).
		            \end{align*}
		            Then demand is:
		            \begin{align*}
			            D_1=\phi_1\sbr{(1-\phi_2)+\phi_2\br{\frac{1}{2}+\frac{P_2-P_1}{2t}}}
		            \end{align*}
		            Hence, the profit is:
		            \begin{align*}
			            \pi_1                          & =(P_1-c)D_1-\frac{a(\Phi_1)^2}{2}                                                                       \\
			            \Longrightarrow\pder{p_1}{P_1} & =\phi_1\sbr{(1-\phi_2)+\phi_2\br{\frac{1}{2}+\frac{P_2-2P_1}{2t}}}=0                                    \\
			            \Longrightarrow P_1            & =\frac{P_2+t}{2}+\frac{1-\phi_2}{\phi_2}                                                                \\
			            \Longrightarrow a\Phi_1        & =\frac{1}{2}P_1\sbr{1-\frac{1}{2}(1+\Phi_1)+\frac{1}{2}(1+\Phi_2)\br{\frac{1}{2}+\frac{P_2-2P_1}{2t}}}.
		            \end{align*}
		      \item Probably: the same intuition still holds. The spillovers in advertisements (in the sense that now ads effectively will reach more
		            people due to word of mouth) will be internalized (i.e., taken into
		            account when setting the advertisement intensity) both by the social planner and in the decentralized equilibrium. Firms will still
		            mix in equilibrium and an ad with the highest price will still capture all the consumer surplus and therefore still be put out at the
		            socially optimal rate. Because of the mixed strategy equilibrium
		            ads with other prices generate the same private benefits as the
		            highest-priced ad and hence will also be put out at the socially
		            optimal rate.
	      \end{enumerate}
	\item Solve the Grossman-Shapiro model in the case that preferences of consumers are
	      given by the Perloff-Salop model described in the first chapter, rather than the
	      Hotelling model. (Do not solve for the welfare optimum: that becomes too complicated.)

	      Solution: In equilibrium (equal prices and ads), firm 1 sells to half the informed consumers:
	      \begin{align*}
		      q_1=\frac{1}{2}(1-(1-\Phi)^2)=\frac{1}{2}\Phi(2-\Phi).
	      \end{align*}
	      Consider when firm 1 defects in both price and advertising:
	      \begin{align*}
		      q_1=\Phi_1\sbr{(1-\Phi_2)+\frac{1}{2}\Phi_2(1-(p_1-p))^2}.
	      \end{align*}
	      Then profits are:
	      \begin{align*}
		      \pi_1=(p_1-c)q_1-\frac{1}{2}a\Phi_1^2.
	      \end{align*}
	      Take FOC:
	      \begin{align*}
		      \pder{\pi_1}{p_1}    & =(p_1-c)\pder{q_1}{p_1}+q_1=0                                    \\
		      \pder{\pi_1}{\Phi_1} & =\sbr{(1-\Phi_2)+\frac{1}{2}\Phi_2(1-(p_1-p))^2}(p_1-c)-a\Phi_1,
	      \end{align*}
	      where:
	      \begin{align*}
		      \pder{q_1}{p_1}=-\Phi_1\Phi_2(1-(p_1-p)).
	      \end{align*}
	      Then we impose symmetry:
	      \begin{align*}
		      -(p-c)\Phi^2+q_1                       & =0                        \\
		      \br{1-\Phi+\frac{1}{2}\Phi}(p-c)-a\Phi & =0                        \\
		      \Longrightarrow                                                    \\
		      p-c-\frac{2-\Phi}{2\Phi}               & =0                        \\
		      \br{1-\frac{1}{2}\Phi}(p-c)-a\Phi      & =0                        \\
		      \Longrightarrow                                                    \\
		      p                                      & = c+\frac{2-\Phi}{2\Phi}  \\
		      \Phi                                   & =\frac{4-8\sqrt{a}}{2-8a}
	      \end{align*}
\end{enumerate}
