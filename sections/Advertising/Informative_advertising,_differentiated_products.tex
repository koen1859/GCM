\subsection{Informative advertising, differentiated products}
Consumers are uniformly distributed on a line of unit length. Each has unit demand and is willing to pay at most $R$, but also faces
transportation costs $t$ per unit of distance. Two firms are located at 0 and 1. Ads are sent randomly. The cost of reaching a fraction
of $\Phi_i$ is denoted by $A(\Phi_i)$. Suppose that $A(\Phi_i)=\frac{a(\Phi_i)^2}{2}$, with $a>\frac{t}{2}$.

There are three kinds of customers. Suppose that firms 1 and 2 inform fractions $\Phi_1$ and $\Phi_2$ of consumers. A fraction
$(1-\Phi_1)(1-\Phi_2)$ is uninformed. A fraction $\Phi_1(1-\Phi_2)$ is captive on firm 1 (and similarly for firm 2). A fraction
$\Phi_1\Phi_2$ is selective. Suppose that the market is always covered, in the sense that a customer who has received at least 1 ad will
always buy. Also assume that the number of selective consumers is sufficiently large that firms are willing to compete for them.
This is true if advertising is not too costly. Firm 1's demand function is:
\begin{align}
	D_1(P_1,P_2,\Phi_1,\Phi_2)=\Phi_1\sbr{(1-\Phi_2)+\Phi_2\frac{(P_2-P_1+t)}{2t}}.
\end{align}
Consider a game in which firms simultaneously choose price and advertising levels. Profits of firm 1 equal:
\begin{align}
	\pi_1 & =(P_1-c)D_1-A(\Phi_1)                                                              \\
	      & =(P_1-c)\Phi_1\sbr{(1-\Phi_2)+\Phi_2\frac{(P_2-P_1+t)}{2t}}-\frac{a(\Phi_1)^2}{2}.
\end{align}
Taking first order condition wrt. $P_1$:
\begin{align}
	\pder{\pi_1}{P_1}=\Phi_1\sbr{(1-\Phi_1)+\Phi_2\frac{P_2-2P_1+c+t}{2t}}=0.
\end{align}
This gives the reaction fuction:
\begin{align}
	P_1=\frac{P_2+t+c}{2}+\frac{1-\Phi_2}{\Phi_2}t.
	\label{eq:ch3foc1}
\end{align}
Now taking first order condition wrt. to $\Phi_1$:
\begin{align}
	a\Phi_1=(P_1-c)\sbr{(1-\Phi_2)+\Phi_2\frac{P_2-P_1+t}{2t}}.
	\label{eq:ch3foc2}
\end{align}
We can impose symmetry and solve (\ref{eq:ch3foc1}) and (\ref{eq:ch3foc2}) simultaneously. From (\ref{eq:ch3foc1}):
\begin{align}
	p^*=c+t\frac{2-\Phi^*}{\Phi^*}.
\end{align}
From (\ref{eq:ch3foc2}):
\begin{align}
	a\Phi & =t\frac{2-\Phi}{\Phi}\br{(1-\Phi)+\frac{\Phi}{2}}     \\
	      & \Longleftrightarrow  a\Phi^2=\br{1-\frac{1}{2}\Phi}^2 \\
	      & \Longrightarrow \sqrt{a}\Phi=\br{1-\frac{1}{2}\Phi}.
\end{align}
Hence:
\begin{align}
	\Phi^*         & =\frac{2}{1+\sqrt{2a/t}}, \\
	\text{and }P^* & =c+\sqrt{2at}.
\end{align}
Equilibrium profits are:
\begin{align}
	\Pi^*=\frac{2a}{\br{1+\sqrt{2a/t}}^2}.
\end{align}
The equilibrium price is higher than in the case without advertising.
The equilibrium advertising level is higher when advertising is less costly, and when
products are more differentiated (that is, when t is higher). Most surprisingly, equilibrium
profits are increasing in the cost of advertising. When $a$ increases, this implies an increase
in costs, but will also lower the equilibrium level of advertising. In this model, the net
effect is positive.

From a welfare point of view, prices are just transfers from consumers to firms. Consumers that are informed about the existence of both firms will
face transportation costs $\frac{t}{4}$ on average. Those that only know the existence of one firm face transportation costs $\frac{t}{2}$
on average. Hence, social welfare is given by:
\begin{align}
	\text{SW}(\Phi)=\Phi^2(v-c-t/4)+2\Phi(1-\Phi)(v-c-t/2)-a\Phi^2.
\end{align}
Maximizing wrt. to $\Phi$ yields:
\begin{align}
	\Phi^S=\frac{2(v-c)-t}{2(v-c)+2a-3t/2}.
\end{align}
This implies that the market equilibrium can either have too much or too little advertising,
depending on the exact parameters of the model.
