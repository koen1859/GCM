\subsection{The Butters Model}
\subsubsection{Setup}
Firms produce a homogeneous product at unit cost $c$. There is free entry of firms and there are $N$ consumers that can only learn a
firm's existence and price by receiving an ad from that firm. Each firm decides how many ads to send, and which price to put in those ads.
Ads are randomly distributed among customers at a cost of $k$ per ad. Consumers have unit demand and are willing to pay at most $R$.
We assume that $R>c+k$. We are interested in the equilibrium in terms of prices and advertising levels, and whether the market provides
too much or too little advertising from a welfare perspective.
\subsubsection{Analysis}
Suppose that $A$ ads are sent. There will be three kinds of consumers. Some customers receive no ads (uninformed). Some consumers only
receive ads from 1 firm (captive) and some consumers receive ads from both firms (selective). These customers buy from the lowest price.

Let $\Phi$ denote the probability that a consumer receives at least 1 ad. The probability that a consumer is uninformed equals
$1-\Phi=\br{1-\frac{1}{N}}^A$. If $N$ is large enough, this is approximately $e^{\frac{A}{N}}$. Thus, if a proportion of $\Phi$ consumers
are to receive at least 1 ad. Then the number of ads that has to be sent equals:
\begin{align}
	A(\Phi)=N\ln{\frac{1}{1-\Phi}}.
	\label{eq:ch3Cost}
\end{align}

A pure strategy equilibrium in prices does not exist. If all other firms charge some price $p\in(k+c,R]$, then this firm has incentive
to slightly undercut. If all other firms charge $p=k+c$ however, then this firm has incentive to charge price $p=R$ and sell to its captive
consumers. The equilibrium has firms mixing on $\sbr{k+c,R}$. For any price below $k+c$ a firm would be better off not sending ads.

There is free entry of firms, implying that equilibrium profits are zero. Denote by $x(P)$ the probability that an ad with price $P$ will
be accepted by the customer receiving it. Then $x(P)$ is the probability that a consumer does not receive an ad with a lower price. This
is decreasing in $P$. Equilibrium requires that for each $P\in[k+c,R]$, we have:
\begin{align}
	(P-c)x(P)-k=0.
\end{align}
This implies that $x(c+k)=1$ and $x(R)=\frac{k}{R-c}$. In equilibrium we also need that a consumer will accept price $R$ exactly equals the
probability that he does not receive any other ad, so $x(R)=1-\Phi$. Equilibrium thus requires:
\begin{align}
	\Phi^*=\frac{1-k}{R-c}.
\end{align}
\subsubsection{Social optimum}
Now consider the amount of advertising that a social planner would choose. Prices are just transfers between consumers and firms, so they will
not affect total welfare. When one additional consumer that was initially uninformed learns about the existence of some firm, the social
benefit is $R-c$. Thus, social benefits are $N\Phi(R-c)$. However, there is a cost of reaching these consumers, which is given by (\ref{eq:ch3Cost}).
The social planner thus maximizes:
\begin{align}
	\underset{\Phi}{\max}\cbr{\Phi N(R-c)-kN\ln{\frac{1}{1-\Phi}} }.
\end{align}
Taking FOC:
\begin{align}
	N(R-c)-\frac{kN}{1-\Phi}=0.
\end{align}
Solving for $\Phi$ yields:
\begin{align}
	\Phi^S=\frac{1-k}{R-c}.
\end{align}
This implies that $\Phi^S=\Phi^*$. Thus, the market provides the socially optimal level of advertising.

This can be understood as follows. Consider the private benefit to a firm of sending
an ad at the price R. This benefit equals $(R-c)$ times the probability that the consumer
receives no other ad. But this is also the social benefit of sending an ad, since the ad
increases social surplus by $R-c$ but only if no other ads are received by the consumer.
Thus, the highest-priced firm appropriates all consumer surplus and steals no business
from rivals. Therefore, it advertises at the socially optimal rate. Now consider an ad at
a price lower than $R$. The private benefits to a firm of sending such an ad are equal to
those of sending an ad at price $P = R$, as expected profits are always zero. This implies
that social benefits also equal private benefits at any price below $R$.
