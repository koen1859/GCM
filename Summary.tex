\documentclass[a4paper,11pt]{article}

\usepackage[english]{babel}
\usepackage{mathtools,amsthm,amssymb,amsmath}
\numberwithin{equation}{section}
\usepackage{booktabs} % For better-looking tables
\usepackage{dcolumn}  % For aligning decimal points in tables
\usepackage{bm}
\usepackage{hyperref}
\usepackage{ marvosym }
\usepackage{eurosym}
\usepackage{bm}
\usepackage{graphicx}
\usepackage{subfig}

\setlength{\parindent}{0pt}     % No paragraph indentation
\setlength{\parskip}{1em}       % Add vertical space between paragraphs

\newcommand{\N}[1]{\mathcal{N}\left(#1\right)}
\newcommand{\Z}{\mathbb{Z}}
\DeclareMathOperator*{\argmin}{arg\,min}
\DeclareMathOperator*{\argmax}{arg\,max}
\newcommand{\abs}[1]{\left\vert#1\right\vert}
\newcommand{\given}{\,\middle|\,}
\newcommand{\Bern}[1]{\mathrm{Bern}(#1)}
\newcommand{\Bin}[1]{\mathrm{Bin}(#1)}
\newcommand{\Exp}[1]{\mathrm{Exp}(#1)}
\newcommand{\FS}[1]{\mathrm{FS}(#1)}
\newcommand{\Geo}[1]{\mathrm{Geo}(#1)}
\newcommand{\Norm}[1]{\mathrm{Norm}(#1)}
\newcommand{\Pois}[1]{\mathrm{Pois}(#1)}
\newcommand{\Unif}[1]{\mathrm{Unif}(#1)}
\newcommand{\E}[1]{\,\mathsf{E}\left[#1\right]}
\newcommand{\EE}[2]{\,\mathsf{E}_{#1}\left[#2\right]}
\newcommand{\V}[1]{\,\mathsf{V}\left[#1\right]}
\newcommand{\cov}[1]{\,\mathsf{Cov}\left[#1\right]}
\renewcommand{\d}[1]{\,\textrm{d}#1}
\newcommand{\1}[1]{\,I_{#1}} % indicator
\renewcommand{\P}[1]{\,\mathbb{P}\left\{#1\right\}}
\newcommand{\fat}[1]{\ThisStyle{\hstretch{1.2}{\ooalign{%
  \kern.46pt$\SavedStyle#1$\cr\kern.33pt$\SavedStyle#1$\cr%
  \kern.2pt$\SavedStyle#1$\cr$\SavedStyle#1$}}}}
\renewcommand{\ln}[1]{\,\mathrm{ln}\left[#1\right]}
\newcommand*{\B}[1]{\ifmmode\bm{#1}\else\textbf{#1}\fi}
\title{GCM Summary and Solutions to exercises}
\author{Koen Stevens}
\date{\today}
\begin{document}
\maketitle
\tableofcontents
\section{Preliminaries}
\subsection{Game Theory}
\subsubsection{Nash Equilibrium}
Each player is choosing the best possible strategy given the strategies chosen by the other players.
\begin{align}
	s^*_i\in \underset{s_i}{\argmax}\,U_i(s^*_1,\cdots,s^*_i,\cdots,s_n^*),\forall i=1,\cdots n.
\end{align}
There is $\in$ instead of $=$ since this Nash Equilibrium does not need to be unique.
\subsubsection{Reaction functions}
A reaction function (or best reply or best response function) gives the best action for a player
given the actions of the other players. The Nash Equilibrium is where all reaction functions intersect.
\subsubsection{Symmetry}
When the game is symmetric, i.e. all players are in the same conditions, have the same reaction function
etcetera, then the players are anonymous (They are indistinguishable from each other except for
name or index). Then all players choose the same strategy $s^*$ in the Nash Equilibrium.
\subsubsection{Sub-game perfect equilibrium}
When players do not move simultaneously, i.e. the players move sequentially, we need to refine
definition of Nash Equilibrium: the sub-game perfect equilibrium requires that the strategy profile
under consideration is not only the equilibrium for the entire game but also for each sub-game.
\subsubsection{Moves of nature}
Many models in IO involve uncertainty. This is often modeled as a “move of nature” in a
multistage game. The moment at which the uncertainty is resolved, is referred to as the
move of nature. Such games can again be solved using backward induction
*** Candidate equilibrium
For ease of exposition, we will often use the concept of a candidate equilibrium. In many
models, it is possible to make an educated guess as to what the equilibrium might be. We
will refer to such an educated guess as a candidate equilibrium. A candidate equilibrium thus is a
strategy profile that may be a (subgame perfect) Nash equilibrium, but for
which we still have to check whether that really is the case. This approach for finding
an equilibrium is often easier than deriving an equilibrium from scratch. Note however
that if it turns out that the candidate equilibrium is a true equilibrium, we still have not
established whether that equilibrium is unique.
*** Mixed strategies
Players are not restricted to always play some action $a_i^*$ in equilibrium. A mixed strategy
equilibrium has each player drawing its action from some probability distribution $F_i(a_i)$,
defined on some domain $A_i$. Given the strategies played by all other players, each player $i$
is indifferent between the actions among which it mixes. Hence, $\mathbb{E}[U_i(a_i)]$ is constant
for all $a_i\in A_i$.
\subsection{Models with differentiated products}
\subsubsection{Hotelling competition}
With horizontal product differentiation different consumers prefer different products. With
vertical product differentiation all consumers agree which product is preferable.

In Hotelling, consumers are normally distributed on line of unit length, the number of consumer
is normalized to 1. Consumers either buy 1 unit of good, or none. Each consumer obtains gross
utility $v$ from consuming the product. There are 2 firms, one located at 0, one located at 1.
Consumer located at $x$ needs to travel distance $x$ to buy from firm 0 and $1-x$ distance to
buy from firm 1. Transportation costs are $t$ per unit of distance. Marginal costs for both firms
are constant and equal to $c$. From this information it can be concluded that in equilibrium both
firms charge the same price since the game is symmetric.
Suppose firm $i$ charges price $P_i$. Then a consumer located at $x$ buys from firm 0 iff
\begin{align}
	v-P_0-tx>v-P_1-t(1-x)
\end{align}
provided that $v-P_0-tx>0$. Assume entire market is covered: in equilibrium prices are s.t.
everyone consumes. This implies that $v>2t$.

Define a consumer $z$ that is indifferent between buying from 0 or 1. Every consumer located at
$x<z$ buys from 0, every consumer $x>z$ buys from 1. This consumer $z$ is located at
\begin{align}
	P_0+tz                & =P_1+t(1-z)                     \\
	\Longleftrightarrow z & =\frac{1}{2}+\frac{P_1-P_0}{2t}
\end{align}
Total sales for firm 0 equal $z$ and for firm 1 $1-z$. When they charge the same price
$z=\frac{1}{2}$. Profits for firm 0 are:
\begin{align}
	\Pi_0 & = (P_0-c)z                                         \\
	      & =(P_0-c)\left(\frac{1}{2}\frac{P_1-P_0}{2t}\right)
\end{align}
Then we maximize wrt $P_0$ to get reaction function:
\begin{align}
	P_0=R_0(P_1)=\frac{1}{2}(c+t+P_1)
\end{align}
And by symmetry:
\begin{align}
	P_1=R_1(P_0)=\frac{1}{2}(c+t+P_0)
\end{align}
Thus, in equilibrium (by equation the two reaction functions):
\begin{align}
	P_0^* & =P_1^*=c+t          \\
	\Pi_0 & =\Pi_1=\frac{1}{2}t
\end{align}
\subsubsection{The circular city: Salop}
Consumers are located uniformly on the edge of a circle with perimeter equal to 1. Consumers wish to buy
1 unit of good, have valuation $v$ and transport cost $t$ per unit of distance. Each firm is
allowed to locate in 1 location. There is a fixed cost of entry $f$ and marginal costs $c$.
Hence, firm $i$ has profit $(p_i-c)D_i-f$ if it enters (where $D_i$) is the demand it faces, and
0 otherwise.

In the first stage, the firms decide simultaneously whether to enter or not. Let $n$ denote the
number of entering firms. The firms are located equidistant from each other on the edge of the
circle.

This can be solved using backward induction. Assume $n$ firms have entered, and suppose all
firms except firm 0 charge price $p$, then firm 0 charges price $R_0(p)$. In equilibrium,
$R_0(p^*)=p^*$. Let firm 0 have location 0. The location of firm 0's right-hand neighbor is
$\frac{1}{n}$ (firm 1). Consider consumers located between firm 0 and 1. If firm 0 charges
price $p_0$, then the consumer $z_{0-1}$ that is indifferent between firm 0 and 1 is:
\begin{align}
	p_0+tz_{0-1}                & =p+t\left(\frac{1}{n}-z_{0-1}\right) \\
	\Longleftrightarrow z_{0-1} & =\frac{1}{2n}+\frac{p-p_0}{2t}
\end{align}
The same holds for the left-hand side of firm 0, hence its profits are:
\begin{align}
	\Pi_0(p_0,p)=2(p_0-c)\left(\frac{1}{2n}+\frac{p-p_0}{2t}\right)
\end{align}
Maximize w.r.t. $p_0$ to find the reaction function:
\begin{align}
	R_0(p)=\frac{1}{2}\left(p+c+\frac{t}{n}\right)
\end{align}
Then we impose symmetry:
\begin{align}
	p^*=c+\frac{t}{n}\text{ with profits }\Pi^*=\frac{t}{n^2}
\end{align}
Now we move back to stage 1. Firms enter the market as long as profits are positive. Hence,
the number of enterers is s.t.
\begin{align}
	\Pi^*-f=\frac{t}{n^2}-f=0\Longleftrightarrow n^* & =\sqrt{\frac{t}{f}} \\
	\Longrightarrow p^*                              & =c+\sqrt{tf}
\end{align}
Consumer's average transportation cost is $\frac{t}{4n}=\frac{\sqrt{tf}}{4}$. When the entry
cost tends to 0 the number of entering firms tends to infinity and the market price tends
to marginal cost.

A social planner would choose $n$ in order to minimize the sum of fixed cost and transportation
costs: $\underset{n}{\min}\left(nf+\frac{t}{4n}\right)
	\Longrightarrow n=\frac{1}{2}\sqrt{\frac{t}{f}}=\frac{1}{2}n^*$, hence the market generates
too many firms. Similar results hold for quadratic transportation costs.

There are three natural extensions that would make the model more realistic: the
introduction of a location choice, the possibility that firms do not enter simultaneously,
and the possibility that a firm locates at several points in the product space.
\subsubsection{Perloff and Salop}
Suppose there are two firms, $A$ and $B$, and a unit mass of customers. Consumer $j$ has a
willingness to pay for the product of firm $i\in\{A,B\}$ that is given by $v+\epsilon_{ij}$,
where $\epsilon_{ij}$ is the realization of a r.v. with cumulative distribution function $F$ on
some interval $[0,\Bar{\epsilon}]$, and continuously differentiable density $f$. Assume $v$ is
high enough that all customers buy in equilibrium. All draws are independent from each other,
$\epsilon_{ij}$ can be interpreted as the match value between product $i$ and customer $j$.
Assume all firms have constant marginal cost equal to $c$.

Firstly, suppose both firms charge same price $p^*$. This will happen in equilibrium since the
game is symmetric. Then a consumer buys from $A$ whenever $\epsilon_A>\epsilon_B$ and from $B$
otherwise. To find the equilibrium we need to analyze what happens if firm $A$ defects from a
candidate equilibrium $p^*$ by setting some different price. Without loss of generality assume $A$
defects to some $p_A>p^*$. Denote $\Delta\equiv p_A-p^*$. Now, a consumer buys from firm $A$ if
$\epsilon_A-\Delta>\epsilon_B$. Assume for simplicity the $\epsilon$ are drawn from uniform
distribution on $[0,1]$. Then, the total sales of firm $A$ from charging price
$p_A=p^*+\Delta$ equal $q_A=\frac{1}{2}(1-\Delta)^2$. Hence the expected profits of firm $A$ are
\begin{align}
	\pi_A=(p_A-c)q_A
\end{align}
Profit maximization requires:
\begin{align}
	\frac{\partial\pi_A}{\partial p_A} & =(p_A-c)\frac{\partial q_A}{\partial p_A}+q_A=0,
	\text{ with }\label{eq:FOC_perloff_salop}                                             \\
	\frac{\partial q_A}{\partial p_A}  & =-(1-\Delta)
\end{align}
Then if we take first order condition and impose symmetry, $\Delta=0$ and $q_A=\frac{1}{2}$, we get:
\begin{align}
	(p_A-c)(-1)+\frac{1}{2} & =0             \\
	\Longleftrightarrow p^* & =c+\frac{1}{2}
\end{align}

In the model description the assumption that $v$ is high enough s.t. each consumer will buy in
equilibrium was made. We thus need that a consumer with $(\epsilon_A,\epsilon_B)=(0,0)$ would still
be willing to buy. In this case, with $p^*=c+\frac{1}{2}$ this implies $v>c+ \frac{1}{2}$

We can generalize the model for any distribution function $F$. Then:
\begin{align}
	q_A & = \int_\Delta^1\left(\int_0^{\epsilon_A-\Delta}f(\epsilon_B)\d\epsilon_B\right)f(\epsilon_A)\d\epsilon_A \\
	    & = \int_\Delta^1F(\epsilon_A-\Delta)f(\epsilon_A)\d\epsilon_A
\end{align}
Using Leibnitz' rule, this implies:
\begin{align}
	\frac{\partial q_A}{\partial p_A}=-\int_\Delta^1f(\epsilon_A-\Delta)f(\epsilon_A)\d\epsilon_A
	\label{eq:pa_leibnitz}
\end{align}
Again we can find the equilibrium price by taking first-order condition (\ref{eq:FOC_perloff_salop})
and using (\ref{eq:pa_leibnitz}). In equilibrium $\Delta=0$ and $q_A= \frac{1}{2}$. This yields:
\begin{align}
	p^*=c+\frac{1}{2\int_0^1f^2(\epsilon)\d\epsilon}
\end{align}
Note that the Perloff-Salop model is similar in spirit to the Hotelling model, in the
sense that each consumer has a certain valuation for each firm’s product. The crucial
difference is that in the Hotelling model, these valuations are perfectly negatively cor-
related: a consumer that has a very high valuation for product A, say, necessarily has
a very low valuation for product B, and vice-versa. In the Perloff-Salop model, these
valuations are independent of each other. A consumer with a high valuation for product
A may very well have a high valuation for product B as well. This also implies that the
Perloff-Salop model can easily be generalized to more than 2 firms.
\subsection{Exercises}
\begin{enumerate}
	\item Consider the following Cournot model. Two firms set quantities. Demand is given by
	      $q=1-p$. Marginal costs are either equal to 0 or 0.4, both with equal probability.
	      Derive the Cournot equilibrium if
	      \begin{enumerate}
		      \item uncertainty is resolved before firms set their quantities.
		      \item uncertainty is resolved after firms set their quantities.
	      \end{enumerate}

	      Solution:
	      \begin{enumerate}
		      \item The game is symmetric, hence in equilibrium both firms set the same
		            price. Let $c=0$. The firms maximize expected profit:
		            \begin{align*}
			            \pi_i=(1-q_1-q_2)q_i\Longrightarrow\frac{\partial\pi_i}{\partial q_i} & =1-q_j-2q_i      \\
			            \overset{\text{FOC}}{\Longrightarrow} q_i                             & =\frac{1-q_j}{2}
		            \end{align*}
		            By symmetry:
		            \begin{align*}
			            q=\frac{1-q}{2}\Longleftrightarrow q & =\frac{1}{3} \\
			            \Longrightarrow p                    & =\frac{1}{3} \\
			            \text{ with profits } \pi_i=\pi      & =\frac{1}{9}
		            \end{align*}
		            Let $c=0.4$. The firms maximize expected profit:
		            \begin{align*}
			            \pi_i=(1-q_1-q_2)q_i-0.4q_i\Longrightarrow\frac{\partial\pi_i}{\partial q_i} & =0.6-q_j-2q_i      \\
			            \overset{\text{FOC}}{\Longrightarrow}q_i                                     & =\frac{0.6-q_j}{2}
		            \end{align*}
		            Again imposing symmetry:
		            \begin{align*}
			            q=\frac{0.6-q}{2}\Longleftrightarrow q & =\frac{1}{5}  \\
			            \Longrightarrow p                      & =\frac{3}{5}  \\
			            \text{ with profits } \pi_i=\pi        & =\frac{3}{25}
		            \end{align*}
		      \item The expected marginal cost is $\mathbb{E}[c]=0.2$. The firms maximize
		            expected profit:
		            \begin{align*}
			            \pi_i                                             & =(1-q_1-q_2)q_i-\mathbb{E}[c]q_i \\
			            \Longrightarrow\frac{\partial\pi_i}{\partial q_i} & =(1-\mathbb{E}[c])-q_j-2q_i      \\
			            \overset{\text{FOC}}{\Longrightarrow}q_i          & =\frac{0.8-q_j}{2}
		            \end{align*}
		            By symmetry:
		            \begin{align*}
			            q=\frac{0.8-q}{2}\Longleftrightarrow q   & =\frac{4}{15}   \\
			            \Longrightarrow p                        & =\frac{7}{15}   \\
			            \text{ with expected profits } \pi_i=\pi & =\frac{28}{225}
		            \end{align*}
	      \end{enumerate}
	\item Consider a Hotelling model. Consumers are uniformly distributed on a line of unit
	      length. Consumers either buy one unit of the good or none at all. Each consumer
	      obtains gross utility $v$ from consuming the product. We have two firms: one is
	      located at 0, the other is located at 1. Marginal costs for both firms are constant
	      and equal to $c$. However, transportation costs are constant: a consumer that has
	      to travel a distance $x$ incurs transport costs $tx^2$. Derive the equilibrium prices.

	      Solution: This game is symmetric hence in equilibrium both firms set the same price.
	      Consider the indifferent consumer with location $z$, then:
	      \begin{align*}
		      v-P_0-tz^2                   & =v-P_1-t(1-z)^2                        \\
		      \Longleftrightarrow P_0+tz^2 & =P_1+t(1-z)^2                          \\
		      \Longrightarrow z            & =\sqrt{\frac{1}{2}+\frac{P_1-P_0}{2t}}
	      \end{align*}
	      Firm 0 its expected profits are:
	      \begin{align*}
		      \pi_0                                             & =(P_0-c)z=(P_0-c)\sqrt{\frac{1}{2}+\frac{P_1-P_0}{2t}}                                       \\
		      \Longrightarrow\frac{\partial\pi_0}{\partial P_0} & =\sqrt{\frac{1}{2}+\frac{P_1-P_0}{2t}}-\frac{P_0-c}{4t\sqrt{\frac{1}{2}+\frac{P_1-P_0}{2t}}} \\
		      \overset{\text{FOC}}{\Longrightarrow}2P_1         & =3P_0-c-2t
	      \end{align*}
	      Imposing symmetry:
	      \begin{align*}
		      2P=3P-c-2t\Longleftrightarrow P=c+2t
	      \end{align*}
	\item Consider the Perloff-Salop model where consumers have a valuation that is uniformly
	      distributed on $[0, 1]$. For simplicity, $c = 0$. Assume however that $v = 0$ such
	      that not all consumers buy in equilibrium. Derive the equilibrium prices.

	      Solution: Let $p_A=p_B=p^*$. Then, a consumer buys from $A$ if:
	      \begin{align*}
		      \epsilon_A-p^*                & >\max\{\epsilon_B-p^*,0\} \\
		      \Longleftrightarrow\epsilon_A & >\max\{\epsilon_B,p^*\}
	      \end{align*}
	      A consumer buys from $B$ if $\epsilon_B>\max\{\epsilon_A,p^*\}$, and does not buy if
	      both $\epsilon_A<p^*$ and $\epsilon_B<p^*$. Firm $A$ faces demand:
	      \begin{align*}
		      q_A & =\P{\epsilon_A>\max\{\epsilon_B,p^*\}}                                                           \\
		          & = \int_0^1\P{\epsilon_A>\max\{\epsilon_B,p^*\}\given\epsilon_B}f(\epsilon_B)\d\epsilon_B         \\
		          & = \int_0^1\P{\epsilon_A>\max\{\epsilon_B,p^*\}}\d\epsilon_B                                      \\
		          & = \int_0^{p^*}\P{\epsilon_A>p^*}\d\epsilon_B + \int_{p^*}^1\P{\epsilon_A>\epsilon_B}\d\epsilon_B \\
		          & = \int_0^{p^*}(1-p^*)\d\epsilon_B + \int_{p^*}^1(1-\epsilon_B)\d\epsilon_B                       \\
		          & = p^*(1-p^*) + (1-p^*) - \frac{1-(p^*)^2}{2}                                                     \\
		          & = \frac{1-(p^*)^2}{2}
	      \end{align*}
	      The expected profit function of firm $A$ is then:
	      \begin{align*}
		      \pi_A                                             & =p^*q_A=p^*\frac{1-(p^*)^2}{2}=\frac{1}{2}p^*-\frac{1}{2}(p^*)^3 \\
		      \Longrightarrow\frac{\partial\pi_A}{\partial p^*} & =\frac{1}{2}-\frac{3}{2}(p^*)^2                                  \\
		      \overset{\text{FOC}}{\Longrightarrow}p^*          & =\frac{1}{\sqrt{3}}
	      \end{align*}
\end{enumerate}
\section{Consumer search}
\end{document}
